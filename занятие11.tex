\section{Фоторезистор}    


\subsection{Устройство фоторезистора}



\subsection{Автоматический уличный фонарь}

\begin{enumerate}
    \itemСоберите схему, представленную на рисунке \ref{}.
    \itemЗамкните выключатель К. Постепенно заслоните фоторезистор и наблюдайте за яркостью свечения лампы.
    \itemПоменяйте местами резистор и фоторезистор и повторите пункт 2.
    \itemЗапишите выводы о возможности применения фоторезистора в конструкции автоматического уличного фонаря.
\end{enumerate}
\begin{figure}[h]
    \centering
    \begin{circuitikz}
        \draw (0,0) node[pnp, tr circle, xscale=-1](Q1){};
        \draw (2,-1) node[npn, tr circle, xscale=-1](Q2){};
        \draw (Q1.B) node[above,xshift=1mm,yshift=-1mm] {\text{Б}};
        \draw (Q2.B) node[above,xshift=1mm,yshift=-5mm] {\text{Б}};
        \draw (Q1.E) node[above, xshift=-3mm, yshift=-5mm] {\text{Э}};
        \draw (Q2.E) node[above,xshift=-3mm, yshift=-1mm] {\text{Э}};
        \draw (Q1.C) node[above,  xshift=-3mm, yshift=-2mm] {\text{К}};
        \draw (Q2.C) node[above,xshift=-3mm, yshift=-5mm] {\text{К}};
        \draw (Q2.B) -- ++(0.25,0)coordinate(p1) {};
        \draw (Q1.C) |- (Q2.E);
        \draw (Q1.B) -| (Q2.C);
        \draw (Q1.E) -| ++(-2,-1) to[battery1] ++(0,-0.5)to[battery1, l=$\mathscr{E}$ ] ++(0,-0.5)to[battery1] ++(0,-0.5)to[battery1] ++(0,-0.5) -- ++(0,-1)coordinate(p2) {};
        \draw (p2) to[normal open switch, l=$K$] ++(2,0) -- ++(2,0)coordinate(p3) {};
        \draw (p3) to[lamp, l=$\text{Л}$] (Q2.E);
        \draw (p3) -| ++(1.09,0) to[ldR, mirror] (p1);
        \draw (Q1.E) -- ++(3.09,0) to[R, l={{{{\rotatebox[origin=c]{-90}{$100~\text{кОм}$}}}}}, invert, a=R] (p1);
    \end{circuitikz}
    \caption{Автоматический уличный фонарь}
    \label{fig:11.1}
\end{figure}


Вывод --- \hrulefill

\hrulefill

\hrulefill



\subsection{Проверка зависимости сопротивления фоторезистора от освещенности}

\begin{enumerate}
    \itemСоберите схему, представленную на рисунке \ref{ris:11.2}.
    \itemПосветите фонариком на фоторезистор. Подкладывая тетрадные листы между фонариком и фоторезистором, наблюдайте за изменением сопротивления фоторезистора. Результаты измерения сопротивления фоторезистора с помощью омметра занесите в таблицу \ref{tab:11.2}.
    \itemЗакройте фоторезистор пальцем и запишите показание фоторезистора в таблицу.  
    \itemЗапишите выводы о зависимости сопротивления фоторезистора от освещенности и о пропускаемой способности света через палец.
\end{enumerate}


\begin{figure}[h]
    \centering
    \begin{circuitikz}[european resistors,
    bigldR/.style={ldR, bipoles/length=2cm}]
    \draw
(0,0) to[bigldR] (0,4) -- (4,4) to[rmeter, t=$\Omega$, bipoles/length=3cm](4,0) -- (0,0)
;
\end{circuitikz}
    \caption{Зависимость сопротивления}
    \label{ris:11.2}
\end{figure}

\begin{table}[h]
\centering
\caption{Зависимость сопротивления фоторезистора от освещенности}
\label{tab:11.2}
\begin{tabular}{|c|c|}
\hline
Количество тетрадных листов, шт & Сопротивление фоторезистора, Ом \\ \hline
0                               &                                 \\ \hline
5                               &                                 \\ \hline
10                              &                                 \\ \hline
Полностью закрыт от света       &                                 \\ \hline
Палец                           &                                 \\ \hline
\end{tabular}
\end{table}

Вывод --- \hrulefill

\hrulefill

\hrulefill



\newpage