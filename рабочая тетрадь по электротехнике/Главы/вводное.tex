\section{Введение}



\setlength\parindent{24pt}
\subsection{Электрический ток}
Электрический ток --- \hrulefill
%Упорядоченное (направленное) движение заряженных частиц

\hrulefill
\\
Условия возникновения электрического тока:
\begin{enumerate}
    \item\hrulefill% Наличие свободных заряженных частиц (носителей заряда).
    \item\hrulefill%Наличие силы, действующей на заряженные частицы (носители заряда) в определённом направлении. $\vec F =q \vec E $
    \item\hrulefill %Замкнутость цепи
\end{enumerate}

\subsection{Сила тока}
Сила тока --- %скалярная физическуая величина, равная отношению заряда ∆q, переносимого через поперечное сечение проводника за промежуток времени ∆t, к величине этого промежутка. Измеряется в амперах [А].
\hrulefill 

\hrulefill

\hrulefill 

Формула силы тока: 

\begin{tikzpicture}
\draw (0,0) rectangle (6,2);
% $I = fraq{\Delta q}{\Delta t}$
\end{tikzpicture}


\subsection{Напряжение}
Напряжение --- %Отношение работы электрического поля по перемещению электрического заряда между двумя точками цепи к этому заряду. Измеряется в вольтах [В].
\hrulefill 

\hrulefill 

\hrulefill 

Формула напряжения: 

\begin{tikzpicture}
\draw (0,0) rectangle (6,2);
% $U = fraq{A}{q}$
\end{tikzpicture}
\subsection{ЭДС}
ЭДС --- \hrulefill
%Электродвижущая сила E равна отношению работы Aст сторонних сил 
%по перемещению положительного заряда q вдоль замкнутой цепи к этому заряду.

\hrulefill
\\
Отличие напряжения от ЭДС:
\begin{enumerate}
    \item \hrulefill

    \hrulefill
    \item \hrulefill

    \hrulefill
\end{enumerate}
%Электродвижущая сила зависит от источника питания, а напряжение зависит от подключенной нагрузки и тока, протекающего по цепи.
%Электродвижущая сила это физическая величина, характеризующая работу сторонних сил неэлектрического происхождения, происходящих в цепях постоянного и переменного тока.

\subsection{Закон Ома для участка цепи}
%\begin{center}
    %\quote{\textit{Знает каждый инженер - сила %тока $''\text{У}~''$~на~$''\text{Эр}~''$!}}
%\end{center}

Закон Ома для участка цепи --- \hrulefill

\hrulefill

\hrulefill

Формула:

\begin{tikzpicture}
\draw (0,0) rectangle (6,2);
% $I = \fraq{U}{R}$
\end{tikzpicture}


\subsection{Закон Ома для полной цепи}
%\begin{center}
    %\quote{\textit{Знает каждый инженер - сила %тока $''\text{У}~''$~на~$''\text{Эр}~''$!}}
%\end{center}

Закон Ома для полной цепи --- \hrulefill

\hrulefill

\hrulefill

Формула:

\begin{tikzpicture}
\draw (0,0) rectangle (6,2);
% $I = \fraq{\mathscr{E}}{R+r}$
\end{tikzpicture}
%
%\begin{table}[h]
 %   \centering
  %  \begin{tabular}{|p{5cm}|p{5cm}|}
   % \hline
    %    Участка цепи &  Полной цепи\\
    %\hline    
     %    & \\  % $I = fraq{U}{R}$
      %   &\\ % $I = fraq{\xi}{R+r}$
       %  &\\
        % &\\
   % \hline   
    %\end{tabular}
%\end{table}


\subsection{Мультиметр}
Мультиметр --- \hrulefill 

\hrulefill 

\hrulefill 

Измерение силы тока

\begin{tikzpicture}
\draw (0,0) rectangle (9,5);
% $U = fraq{A}{q}$
\end{tikzpicture}
\\
Мультиметр подключается --- \hrulefill
\\
Щупы вставляются в клемы при токе порядка $100 mA$:

Черный -- \hrulefill

Красный -- \hrulefill
\\
При токе порядка $1 A$:

Черный -- \hrulefill

Красный -- \hrulefill

Измерение напряжения

\begin{tikzpicture}
\draw (0,0) rectangle (9,5);
% $U = fraq{A}{q}$
\end{tikzpicture}
\\
Мультиметр подключается -- \hrulefill
\\
Щупы вставляются в клемы:

Черный -- \hrulefill

Красный -- \hrulefill


\subsection{Схематическое обозначение электронных компонентов}

\begin{circuitikz}[european]
\centering
\draw (0,0) -- (3,0); 				\draw (6,0) to[lamp] (9,0); 		\draw (12,0) to[empty led] (15,0);
\draw (0,-2) to[normal open switch] (3,-2); 	\draw (6,-2) to[buzzer] (9,-2); 		\draw(12,-2) to[C] (15,-2);
\draw (0,-4)  to[push button](3,-4); 		\draw (6,-4) to[R] (9,-4);			\draw (12,-4) to[cute inductor] (15,-4);
\draw (0,-6) to[reed](3,-6); 			\draw (6,-6) to[pR] (9,-6);		\draw (12,-6) to[mic] (15,-6);
\draw (0,-8) to[battery1](3,-8); 			\draw (6,-8) to[empty diode] (9,-8);	\draw(12,-8)to[loudspeaker](15,-8);
\end{circuitikz}


\subsection{Первая цепь. Измерение силы тока и напряжения}


\begin{enumerate}
    \item Соберите схему, изображенную на рисунке %\ref{fig:fcircuit}.
    \item Измерьте силу тока и напряжение на лампе.
    \item Рассчитайте сопротивление лампы по закону Ома для участка цепи.
    \item Занесите значения в таблицу %\ref{tab:0.1}.
    \item Предположите и запишите в вывод, что нужно измерить, чтобы рассчитать сопротивление лампы по закону Ома для полной цепи.
\end{enumerate}

\begin{figure}[h]
    \centering
    \begin{circuitikz} \draw
	(0,0) to[lamp=$\text{Л}$, fill=yellow] (0,4)
	  -- (4,4) to[battery1, l= $\mathscr{E}$] (4,0)
	  to[normal open switch=$K$, mirror](0,0)
	;
    \end{circuitikz}
    \caption{Измерение силы тока и напряжения}
    %\label{fig:fcircuit}
\end{figure}



\begin{table}[h]
    \caption{Измерение силы тока и напряжения.}
    %\label{tab:0.1}
    \centering
    \begin{tabular}{|c|c|с|}
    \hline
       Сила тока I, А  & Напряжение U, В & Сопротивление R, Ом \\
    \hline
        & & \\
        & & \\
    \hline     
    \end{tabular}
    
\end{table}


Вывод --- \hrulefill

\hrulefill

\hrulefill

\newpage