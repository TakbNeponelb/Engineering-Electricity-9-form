\section{Источники питания}


\subsection{Последовательное подключение батарей.}

\begin{enumerate}
    \item Соберите по очереди схемы, изображенные на рисунке \ref{ris:image1}.
    \item Сравните яркость лампы во всех трех случаях.
    \item Укажите направление тока для каждой цепи.
    \item Измерьте напряжение на лампе, напряжение на батарее, силу тока в цепи в каждом случае и занесите полученные значения в таблицу \ref{tab:1.1}.
    \item Запишите вывод о полученных результатах.
\end{enumerate}

\begin{figure}[h]
\begin{minipage}[h]{0.3\linewidth}
\center{
\begin{circuitikz} \draw
(0,0) to[lamp=$\text{Л}$, fill=yellow] (0,3)
  -- (3,3) to[battery, l= $\mathscr{E}$]  (3,0)
  to[normal open switch=$K$, mirror](0,0)
;
\end{circuitikz}
\\ а.}
\end{minipage}
\hfill
\begin{minipage}[h]{0.3\linewidth}
\center{
\begin{circuitikz} \draw
(0,0) to[lamp=$\text{Л}$, fill=yellow] (0,3)
  -- (3,3) -- (3,2.1) to[battery] (3,1.5) to[battery, l= $\mathscr{E}$] (3,0.9) -- (3,0)
  to[normal open switch=$K$, mirror](0,0)
;
\end{circuitikz} \\ б.}
\end{minipage}
\hfill
\begin{minipage}[h]{0.3\linewidth}
\center{
\begin{circuitikz} \draw
(0,0) to[lamp=$\text{Л}$, fill=yellow] (0,3)
  -- (3,3) -- (3,2.1) to[battery] (3,1.5) to[battery, l= $\mathscr{E}$, invert] (3,0.9) -- (3,0)
  to[normal open switch=$K$, mirror](0,0)
;
\end{circuitikz} \\ в.}
\end{minipage}
\caption{Последовательное подключение источников тока.}
\label{ris:image1}
\end{figure}

\begin{table}[h]
    \caption{Последовательное подключение}
    \centering
    \begin{tabular}{|c|c|c|c|}
    
    \hline
    Схема & {ЭДС $\mathscr{E}$, В} & {Сила тока I, А} & {Напряжение U, В} \\
    \hline    
       a.  & & & \\ 
          \hline 
        б. & & & \\
           \hline 
        в. & & & \\ 
    \hline   
    \end{tabular}
     \label{tab:1.1}
\end{table}

Вывод --- \hrulefill 

\hrulefill 

\hrulefill 


\subsection{Параллельное подключение батарей.}

\begin{enumerate}
    \item Соберите по очереди схемы, изображенные на рисунке \ref{ris:parallel}.
    \itemДля каждого случая выполните следующее задание.
    \begin{enumerate}
    \item Замкните сначала один ключ, пронаблюдайте яркость лампы.
    \item Измерьте силу тока в цепи и напряжение на лампе. Занесите полученные данные в таблицу.
    \item Замкните оба ключа, сравните яркость с предыдущим случаем.
    \item Измерьте силу тока в цепи и напряжение на лампе. Занесите полученные данные в таблицу \ref{tab:1.2}.
    \end{enumerate}
    \itemЗапишите вывод о полученных результатах. 
\end{enumerate}

\newpage

\begin{figure}[h]
\begin{minipage}[h]{0.5\linewidth}
\center{\begin{circuitikz} \draw
(0,0) to[normal open switch=$K_1$] (0,2) to[battery, l=$\mathscr{E}_1$] (0,4) --
  (4,4) to[lamp=$\text{Л}$,fill=yellow]  (4,0)
  --(0,0)
  (2,0) to[normal open switch=$K_2$] (2,2) to[battery, l=$\mathscr{E}_2$] (2,4)
;
\end{circuitikz}\\ а.}
\end{minipage}
\hfill
\begin{minipage}[h]{0.5\linewidth}
\center{\begin{circuitikz} \draw
(0,0) to[normal open switch=$K_1$] (0,2) to[battery, l=$\mathscr{E}_1$] (0,4) --
  (4,4) to[lamp=$\text{Л}$,fill=yellow]  (4,0)
  --(0,0)
  (2,0) to[normal open switch=$K_2$] (2,2) to[battery, l=$\mathscr{E}_2$, invert] (2,4)
;
\end{circuitikz} \\ б.}
\end{minipage}
\caption{Параллельное подключение источников тока.}
\label{ris:parallel}
\end{figure}

\begin{table}[h]
 \caption{Параллельное подключение источников}
    \centering
    \begin{tabular}{|c|c|c|c|}
        \hline
         Схема& Положение ключей & {Сила тока I, А} & {Напряжение U, В} \\
        \hline  
       & замкнут 1 ключ  &   &   \\
        \cline{2-4}
        \raisebox{1.5ex}[0cm][0cm]{2.а.} 
        &замкнуто 2 ключа  &   &   \\
        \hline
         & замкнут 1 ключ  &   &   \\
        \cline{2-4}
        \raisebox{1.5ex}[0cm][0cm]{2.б.}
        &замкнут 2 ключа  &   &   \\
        \hline
    \end{tabular}
    \label{tab:1.2}
\end{table}

Вывод --- \hrulefill 

\hrulefill 

\hrulefill

\subsection{Внутреннее сопротивление батареи.}

\begin{enumerate}
    \item Соберите схему, изображенную на рисунке \ref{ris:1.4}.
    \item Замкните ключ
    \itemИзмерьте ЭДС батареи и силу тока в цепи
    \itemРассчитайте внутреннее сопротивление $r$ по формуле:
    \begin{center}
    \Large $r=\frac{\mathscr{E}}{I}$    
    \end{center}
    
    \item Занесите значение в таблицу \ref{tab:1.4}.
    \item Сделайте вывод о возможности пренебрежения внутреннего сопротивления, по сравнению с резисторами, входящими комплект.
\end{enumerate}



\begin{figure}[h]
\center{\begin{circuitikz} \draw
(0,0) to[rmeter, t=A] (0,4) to[normal open switch=$K$] (4,4) --
  (4,4) to[battery1, l={$\mathscr{E}$, $r$}]  (4,0)
  --(0,0)
(4,3) -- (2,3) to[rmeter, t=V] (2,1) -- (4,1)
 
;
\end{circuitikz}}
\caption{Внутреннее сопротивление источника.}
\label{ris:1.4}
\end{figure}

\newpage

\begin{table}[h]
    \caption{Внутреннее сопротивление}
    \centering
    \begin{tabular}{|c|c|c|}
    \hline
    \multicolumn{1}{|c|}{\begin{tabular}[c]{@{}c@{}}ЭДС батареи\\ $\mathscr{E}$, В\end{tabular}} & \multicolumn{1}{c|}{\begin{tabular}[c]{@{}c@{}}Сила тока\\ короткого замыкания\\$I_\text{кз}$, А\end{tabular}} & \multicolumn{1}{c|}{\begin{tabular}[c]{@{}c@{}}Внешнее \\ сопротивление r, Ом\end{tabular}} \\ \hline
    
   &   &    \\
    &   &    \\\hline
   
    \end{tabular}
    
     \label{tab:1.4}
\end{table}

Вывод --- \hrulefill 

\hrulefill 

\hrulefill 

\subsection{*Порог включения}

\begin{enumerate}
\item Придумайте и соберите схему для исследования разной силы тока и напряжения работающей лампы.
\item Нарисуйте вашу схему в прямоугольнике ниже. 
\item Определите минимальное значение силы тока и напряжения работающей лампы. Занесите значения в таблицу \ref{tab:1.5}.
\item Опишите в выводе принцип работы вашей схемы и метод нахождения и расчета значений.
\end{enumerate}


\begin{figure}[h]
\centering
\begin{tikzpicture}
	\draw
	(0,0) rectangle (14,6)
	;
\end{tikzpicture}
\caption{Порог включения}
\label{fig:2.5}
\end{figure}

\begin{table}[h]
\centering
\caption{Порог включения}
\label{tab:1.5}
\begin{tabular}{|c|c|}
\hline
$I_{min}$, А & $U_{min}$, В \\
\hline
& \\
& \\
\hline
\end{tabular}

\end{table}

Вывод --- \hrulefill

\hrulefill

\hrulefill

\hrulefill

\newpage